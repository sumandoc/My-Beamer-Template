\documentclass[12pt]{article}

\usepackage{amsmath}

\usepackage{amsfonts}
\usepackage{amssymb}
\usepackage{graphicx}


\usepackage[colorlinks = true,
linkcolor = blue,
urlcolor  = blue,
citecolor = blue,
anchorcolor = blue]{hyperref}

\renewcommand\theequation{{\color{blue}\arabic{equation}}}

\title{Sample size for various studies}
\author{Dr Suman Khanal, MS\footnote{\textbf{\copyright} Dr Suman Khanal \textbf{2017}}\\ Kathmandu}
\date{July 1, 2017\footnote{Made with $\heartsuit$ with \href{https://www.latex-project.org/}{\LaTeXe}}}

\begin{document}
\maketitle


\section*{Sample size}	

\section{Cross sectional studies/surveys}

\subsection{Qualitative variable}
	Suppose an epidemiologist want to know proportion of 
	children who are hypertensive in a population then this 
	formula should be used as proportion is a qualitative 
	variable.
	
\begin{equation}
Sample size =  \dfrac{1.96\times 1.96\times p(1-p)}{d^2}
\end{equation}
where,\\ 
p  =  Expected  proportion  in  population  based  on  
previous studies or pilot studies.\\
d  =  Absolute  error  or  precision  –  Has  to  be  decided  
by researcher.

\subsection{Quantitative variable}
Suppose the same researcher is interested in knowing 
average systolic blood pressure of children of the same 
city then below mentioned formula should be used as 
blood pressure is a quantitative variable.

\begin{equation}
Sample size=\dfrac{1.96\times 1.96\times {SD}^2}{d^2}
\end{equation}
where,\\
SD = Standard deviation of variable. Value of standard 
deviation can be taken from previously done study or 
through pilot study.\\
d  =  Absolute  error  or  precision  as  mentioned previously


\section{Case-control studies}

\subsection{Qualitative variable}
Suppose  a  researcher  want  to  see  the  link  between  
childhood  sexual  abuses  with  psychiatric  disorder  in  
adulthood.  He  will  take  a  sample  of  adult  persons  
with psychiatric disorder and will take another sample 
of normal adults having no psychiatric disorders. He 
will then go retrospectively to see history of childhood 
sexual abuse in both groups. Exposure to both groups 
will be compared and odds ratio will be calculated. Here 
number of people exposed to childhood sexual abuse 
is qualitative variable hence this formula will be used 
for such type of design.

\begin{equation}
Sample size =\dfrac{r+1}{r}\dfrac{(p^*)(1-p^*)(Z_\beta+Z_{\alpha/2})^2}{(p_1-p_2)^2}
\end{equation}
where,\\
r = Ratio  of  control  to  cases,  1  for  equal  number  of  
case and control\\
p*  =  Average  proportion  exposed  =  (proportion  of  
exposed cases + proportion of control exposed)/2\\
$Z_\beta$ =  Standard  normal  variate  for  power  =  for  80\%  
power it is 0.84 and for 90\% is 1.28. Researcher 
has to select power for the study.\\
$Z_{\alpha/2}$ = Standard normal variate for level of significance 
as mentioned in previous section.\\
$p_1-p_2$= Effect size or different in proportion expected 
based on previous studies. $p_1$ is proportion in cases and 
$p_2$ is proportion in control.

\subsection{Quantitative variable}

Suppose  a  researcher  wants  to  see  the  association  
between birth weight and diabetes in adulthood. The 
birth weight being a quantitative data, the researcher will 
select one group i.e. cases that will be diabetic adults and 
other group i.e. control will be non‑diabetic adults. Both 
groups will be traced back for data regarding childhood 
weight. The formula for sample size calculation is:

\begin{equation}
Sample size=\dfrac{r+1}{r}\dfrac{SD^2(Z_\beta+Z_{\alpha/2})^2}{d^2}
\end{equation}
where,\\
SD = Standard deviation = researcher can take value 
from previously published studies\\
d  =  Expected  mean  difference  between  case  and  
control (may be based on previously published studies.)\\
r = Ratio  of  control  to  cases,  1  for  equal  number  of  
case and control\\
$Z_\beta$ =  Standard  normal  variate  for  power  =  for  80\%  
power it is 0.84 and for 90\% is 1.28. Researcher 
has to select power for the study.\\
$Z_{\alpha/2}$ = Standard normal variate for level of significance 
as mentioned in previous section.


\section{Cohort studies}


\subsection{title}


\begin{equation}
Sample size=\dfrac{Z_\alpha\sqrt{1+\dfrac{1}{m}p^*(1-p^*)}+Z_\beta\sqrt{\dfrac{p_1(1-p_1)}{m}+p_2(1-p_2)}}{(p_1-p_2)^2}
\end{equation}

$p^*=\dfrac{p_2+mp_1}{m+1}$

$p_1$=control group

$p_2$=experimental group\\



\textbf{\large These formulas are for independent design studies. Not for matched case control and cohort studies.}
	
	
	
	
	
	
	
	
	
\end{document}
