\documentclass{article}

\usepackage[latin1]{inputenc}
\usepackage{tikz}
\usepackage{amsmath, amssymb}
\usetikzlibrary{shapes,arrows}
\begin{document}
%\pagestyle{empty}


% Define block styles
\tikzstyle{decision} = [diamond, draw, fill=blue!20, 
    text width=4.5em, text badly centered, node distance=3cm, inner sep=0pt]
\tikzstyle{block} = [rectangle, draw, fill=blue!20, 
    text width=7em, text centered, rounded corners, minimum height=4em]
\tikzstyle{line} = [draw, -latex']


    
 \begin{figure}
 \centering
  
    
\begin{tikzpicture}[node distance = 2cm, auto]
    % Place nodes
    \node [block] (init) {Parathyroid Adenoma};
    
    \node [block, below of=init] (identify) {High levels of PTH, Ca2+};
    \node [block, below of=identify] (evaluate) {\textbf{Other} parathyroids are \textbf{suppressed}};
    \node [decision, below of=evaluate] (decide) {Once adenoma removed};
    \node [block, below of=decide, node distance=3cm] (stop) {PTH, Ca2+ falls};
	\node [block, below of=stop, node distance=3cm] (final) {PTH \textbf{gradually increases to normal} once the \textbf{suppressed} parathyroid pick up};
    % Draw edges
    \path [line] (init) -- (identify);
    \path [line] (identify) -- (evaluate);
    \path [line] (evaluate) -- (decide);
    
    \path [line] (decide) -- (stop);
	\path [line] (stop) -- (final);
	    
\end{tikzpicture}
	\caption{Why PTH level will remain decreased initially after removal of adenoma.}
	
\end{figure}  


\end{document}
