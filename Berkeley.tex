\documentclass[]{beamer}
\usepackage{beamerthemesplit}
\usepackage{amsmath}

\usepackage{hyperref}

\usetheme{Berkeley}
%\usecolortheme{beaver}


\title[Electrosurgery]{ELECTROSURGERY}
\author[Dr Suman]{Dr Suman *}
\date{\today}
\institute[Department of Surgery]{* Teaching Hospital}

\logo{\includegraphics[height=10mm,width=10mm]{TUlogo.png}}

\begin{document}
\maketitle

\section{Introduction}

	
\begin{frame}{Introduction}
	\begin{itemize}
		\item Electrosurgery remains an indispensable tool for the modern dermatologist. 
		\item Despite the increasing use of
		ablative lasers, the humble electrosurgical
		machine remains important because of its
		portability and inexpensiveness.
		\item  Is an effective modality for the
		treatment of a variety of skin lesions.
	\end{itemize}

\end{frame}

\section{History}

\begin{frame}
\begin{itemize}
	\item First true use of electrosurgery dates back to the last decade of the 19th century when Joseph River, a Parisian physician, used an arcing current to treat a cutaneous carcinoma.
	\item William T Bovie developed the modern electrosurgical device in the 1920’s. This was used for the first time by Dr Harvey Cushing at Peter Bent Brigham Hospital in Boston, on October 1, 1926, to successfully remove an enlarging, vascular myeloma from the head.
\end{itemize}
\end{frame}


\section{Fundamentals}
\begin{frame}{Electrocautery vs Electrosurgery}

\begin{itemize}
	\item Frequently confused but are quite different in terms of both tools used and method of application.
	\item \textit{Electrocautery} uses electrical current to heat a metal wire that is then applied to the target tissue in order to burn or coagulate the specific area of tissue.
	\begin{itemize}
		\item Not used to pass the current through tissue, but rather applied directly onto the targeted area of treatment. 
		\item Heat is generated in resistant metal wire which is used as an electrode. This hot electrode is then placed directly onto the treatment area destroying that specific tissue. 
		\item Used in superficial situations by dermatologists, ophthalmologists, plastic surgeons
	\end{itemize} 
\end{itemize}
\end{frame}


\begin{frame}
\begin{itemize}
	\item \textit{Electrosurgery} passes electrical current through tissue. 
	\item The electricity used is a form of alternating current similar to the that used to generate radio waves. The typical frequency is quite high,approx 500,000 cycles per second.
	\item The heat is created by the resistance of the tissue to the electrical current and the tools used to apply the current are electrodes and includes  blades, round ball, needle and loop configurations.
	\item Can be used to cut, coagulate , or even to fuse tissue.
		
	
	\end{itemize}
	
 \end{frame}

\begin{frame}
\begin{itemize}
	\item \textit{Electrocautery} devices are usually small, battery operated,   devices which use physical heat to destroy the targeted tissues or cause a specific and desired effect. 
	\item \textit{Electosurgery} devices are more sophisticated radio-wave generators that pass modified electrical current through the target tissues to achieve the desired surgical result.
	
\end{itemize}
\end{frame}
	
	
	
	
\end{document}
